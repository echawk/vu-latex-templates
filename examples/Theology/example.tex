\documentclass[letterpaper]{turabian-researchpaper}
\usepackage[utf8]{inputenc}
\usepackage[autostyle, english = american]{csquotes}
%\MakeOuterQuote{"} % uncomment this line if you want to have 'smart' quotes like in word
\usepackage[pass, letterpaper]{geometry}
\usepackage[american]{babel}
\usepackage[backend=biber,notes,style=sbl]{biblatex} % Can also include 'annotation' within the '[]'
\addbibresource{CC.bib} % Add our bibliography file
\usepackage{hyperref} %Allows links to be clicked
\usepackage{wrapfig}
\usepackage{setspace} %\doublespace
\usepackage{fancyhdr} % allows the setting of a custom header
\pagestyle{fancy} %fancyhdr package
\fancyhf{} %clears out prev header/footer
\rhead{\hfill \thepage
	\\ \hfill John Doe
	\\ \hfill THEO200 }
\renewcommand\headrulewidth{0pt} %gets rid of the bar > sets it to 0pt
% this is an example of how to define a custom command, in this case,
% we are defining a new command '\ccparencite' that can be used to parenthetically cite sources.
\newcommand{\ccparencite}[1]{(\citeauthor{#1})}

% here is a command that just prints "Valparaiso University"
\newcommand{\vu}{Valparaiso University}

\title{My Very Important Paper (Due Saturday)}
\subtitle{}
\author{John Doe}
\course{THEO200}
\date{\today}

\begin{document}

\maketitle
\doublespacing

\begin{center}
	\textbf{My Very Important Paper (Due Saturday)}
\end{center}


This is an example document to showcase how to do different things in \LaTeX.
Let's make a footnote\footnote{It shows up!}. Here is a parenthetical citation\parencite{kant}.
That is \textit{technically} the correct way to cite things paranthetically in Chicago style, so let's
do a footnote style citation\autocite{akut, 34}. Now that looks \textbf{better}.

\section{Here's a section heading!}


Blah blah blah. Here's how to cite with your own custom citation command\ccparencite{kant, 23}.

\subsection{Here's a subsection\ldots}
\subsubsection{And a sub-subsection...}

Etc. Etc. \vu\vu\vu

Let's cite Akutagawa again\autocite{akut}, because the citation will be different.

Now let's try a parenthetical citation with a page number\parencite{mill, 54}.

This is how you write ``a quote''\parencite*{akut, 84}.
Now it's time to sign the honor code. So here's an example of
how to ``sign'' your paper.

\noindent\rule{\textwidth}{1pt} % makes a line that goes across the page.
\begin{center}
	\textit{I have not given or received nor have I tolerated others' use of unauthorized aid.}
	\newline
	\textit{- John Doe}
\end{center}

\newpage
%\nocite{*} % Removing the comment will print out everything in the .bib file
\printbibliography

\end{document}
